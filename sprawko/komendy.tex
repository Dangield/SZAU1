\chapter{Skrypty}
	\label{ch:skrypty}
	W celu uzyskania wykresów zamieszczonych w sprawozdaniu należy wywołać podane skrypty:
	\section{Zadanie 1}
	\begin{itemize}
		\item badanie modelu:\\
		$z1\_model$
		\item porównanie modelu nieliniowego i liniowego:\\
		$z1\_modellin$
		\item odpowiedź skokowa w punkcie pracy:\\
		$z1\_step(16, true)$
		\item przebiegi regulacji dla DMC:\\
		$z1\_dmc(400, 400, 400, 400, 100, true)$\\
		$z1\_dmc(400, 400, 400, 400, 1000, true)$\\
		$z1\_dmc(400, 400, 400, 400, 10000, true)$\\	
	\end{itemize}
	\section{Zadanie 2}
	\begin{itemize}
		\item obiekty rozmyte:\\
		$z2\_modelroz(2, true)$\\
		$z2\_modelroz(3, true)$\\
		$z2\_modelroz(4, true)$\\
		$z2\_modelroz(5, true)$\\
		\item dmc rozmyty, lambda 2000, oryginalne zbiory rozmyte:\\
		$z2\_dmcroz(400, 400, 400, 400, 2000,[],[], true)$
		\item dmc rozmyty, lambda 2000, nowe zbiory rozmyte:\\
		$z2\_dmcroz(400, 400, 400, 400, 2000,10,[8.5000   17.5000   24.5000   33.5000], true)$
		\item dmc rozmyty, lambda 100, nowe zbiory rozmyte:\\
		$z2\_dmcroz(400, 400, 400, 400, 100,10,[8.5000   17.5000   24.5000   33.5000], true)$
		\item dmc rozmyty, lambda 100, nowe zbiory rozmyte, skrócone horyzonty:\\
		$z2\_dmcroz(200, 150, 100, 200, 100,10,[8.5000   17.5000   24.5000   33.5000], true)$
	\end{itemize}
	\section{Zadanie 3}
	\begin{itemize}
		\item uruchomienie regulatora SL:\\
		$z3\_sl(200, 150, 100, 200, 100, true)$
	\end{itemize}
	
	\section{Zadanie 4}
	\begin{itemize}
		\item uruchomienie regulatora NPL:\\
		$z4\_npl(200, 150, 100, 200, 100, true)$
	\end{itemize}