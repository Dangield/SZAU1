\chapter{Numeryczny rozmyty regulator SL}
	Rozmyty regulator typu SL jest w pewnym sensie podobny do rozmytego regulatora DMC, z dokładnością do tego, kiedy następuje rozmycie. Otóż w regulatorze SL rozmywana jest nie końcowa zmiana sterowania, a odpowiedzi skokowe regulatorów. W każdym kroku regulacji na podstawie poszczególnych odpowiedzi skokowych oraz aktualnych wag obliczana jest wspólna odpowiedź skokowa, a następnie następuje obliczane są nowe macierze $M$, $M^P$ i $M^{zP}$ na postawie których obliczana jest rozmyta odpowiedź swobodna obiektu, wykorzystywana do regulacji.
	
	Ponieważ wspomniany rozmyty algorytm SL miał zostać zaimplementowany w formie numerycznej zmienił się także sposób obliczania zmiany sterowania. Jest ona obliczana poprzez rozwiązanie problemu optymalizacji liniowo-kwadratowej przedstawionego wzorem \ref{eq:sl}, w którym $\tilde{y}^0$ oznacza rozmytą odpowiedź swobodną, a $M_k$ rozmytą macierz $M$. Problem ten w naszej implementacji rozwiązujemy za pomocą wbudowanej w Matlab funkcji minimalizacji $fmincon()$. W funkcji tej bierzemy także pod uwagę ograniczenia co do wielkości maksymalnej oraz minimalnej sterowania. Ograniczenia odnośnie maksymalnej zmiany sterowania w kroku po konsultacji z prowadzącym nie zostały wzięte pod uwagę.
	
	Podczas sterowania regulatorem SL długości horyzontów, lambda oraz kształt funkcji przynależności przyjęte zostały takie same jak dla ostatniego przebiegu rozmytego regulatora DMC w zadaniu 2, czyli lambda równa 100, D równe 200, N równe 150, Nu równe 100, Dz równe 200 oraz funkcje przynależności tak jak na wykresie \ref{rys:dmcprzyn}.
	
	\begin{equation}
		min_{\Delta u}\{(y^{zad}-\tilde{y}^0-M_k*\Delta U)^T(y^{zad}-\tilde{y}^0-M_k*\Delta U)-\lambda \Delta u^T*\Delta u\}
		\label{eq:sl}
	\end{equation}